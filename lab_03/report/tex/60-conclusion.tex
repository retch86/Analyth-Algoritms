\part*{ЗАКЛЮЧЕНИЕ}
\addcontentsline{toc}{part}{\textbf{ЗАКЛЮЧЕНИЕ}}
В результате замеров времени выполнения реализаций различных алгоритмов было выявлено, что рекурсивная реализация быстрой сортировки для лучшего случая выполняется медленнее остальных рассмотренных алгоритмов для 800 элементов массива в среднем в 2 раза. В дальнейшем этот отрыв увеличивается. Блинная и сортировка выбором имеют разницу во времени выполнения в среднем на 1-3\%. 

При случайной генерации массива быстрая сортировка оправдывает свое название. Разница с блинной сортировкой составляет в среднем 32,5 раза для 600 элементов и 48 раз для 900. Разница сортировки выбором и быстрой составляет 13,7 раз для 600 элементов и 19,3 раз для 900. Время выполнения сортировок начинается отличаться уже при 200 элементов массива.

Для худшего случая быстрая сортировка вновь проигрывает по времени выполнения блинной и быстрой в среднем в 1,7 раз.

В результате теоретической оценки алгоритмов по памяти показано, что алгоритм сортировки выбором является наименее ресурсозатратным по сравнению с блинной и быстрой.
Алгоритм быстрой сортировки, напротив, требует больше всего памяти, что объясняется тем, что алгоритм рекурсивный, и на каждый вызов функции требуется выделение памяти на стеке для сохранения информации, связанной с этим вызовом.