\chapter{Технологический раздел}
В разделе рассматриваются средства реализации, а также приводятся листинги алгоритмов сортировки <<блинная>>, <<быстрая>>, <<выбором>>.

\section{Средства реализации}
В работе для реализации алгоритмов был выбран язык программирования Python \cite{python}. В нем присутствуют библиотека time \cite{process_time} для замера процессорного времени process\_time(), а также для замера используемой памяти при помощи декоратора profiler \cite{profiler}.

\section{Модули программы}
\subsection{Блинная сортировка}
Реализация блинной сортировки приведена на листинге \ref{lst:pancake_sort}.
\lstinputlisting[label=lst:pancake_sort, caption=Блинная сортировка, basicstyle=\footnotesize, numbers=none]{lst/pancake_sort.py}
Блинная сортировка использует функцию переворачивания элементов flip() (листинг \ref{lst:flip}) и нахождение индекса первого максимального элемента \text{(листинг \ref{lst:max_index})}.
\lstinputlisting[label=lst:flip, caption=Переворачивание элементов массива, basicstyle=\footnotesize, numbers=none]{lst/flip.py}
\newpage
\lstinputlisting[label=lst:max_index, caption=Нахождение первого максимального элемента, basicstyle=\footnotesize, numbers=none]{lst/max_index.py}
\subsection{Быстрая сортировка}
Реализация быстрой сортировки приведена на листинге \ref{lst:quick_sort}.
\lstinputlisting[label=lst:quick_sort, caption=Быстрая сортировка, basicstyle=\footnotesize, numbers=none]{lst/quick_sort.py}
Быстрая сортировка использует функцию partition() разделения элементов массива на две части с упорядочиванием значений относительно опорного элемента (листинг \ref{lst:partition}).
\lstinputlisting[label=lst:partition, caption=Разделение элементов массива на две части с обменом, basicstyle=\footnotesize, numbers=none]{lst/partition.py}
\subsection{Сортировка выбором}
Реализация сортировки выбором приведена на листинге \ref{lst:select_sort}.
\lstinputlisting[label=lst:select_sort, caption=Сортировка выбором, basicstyle=\footnotesize, numbers=none]{lst/select_sort.py}

\section{Тестирование}
Для тестирования используется метод черного ящика. В данном разделе приведена таблица \ref{table:ref2}, в которой указаны классы эквивалентностей тестов: \\

\begin{table}[H]
	\centering
	\captionsetup{singlelinecheck = false, justification=raggedleft}
	\caption{Тесты}
	\label{table:ref2}
	\begin{tabular}{|c|c|c|c|c|c|}
		\hline
		\multirow{3}{*}{№} & \multirow{3}{*}{Описание теста} & \multirow{3}{*}{Вход} & \multicolumn{3}{c|}{Результат}\\ \cline{4-6}
		&                &          &Блинная          &Быстрая  &Выбором	\\
		\hline
		1& Один элемент  &  1      &    1      &   1         &  1 						\\ \hline
		\multirow{2}{*}{2}& \multirow{2}{*}{Отсортированный} & \multirow{2}{*}{1,2,3,4,5} & \multirow{2}{*}{1,2,3,4,5} & \multirow{2}{*}{1,2,3,4,5}   &  \multirow{2}{*}{1,2,3,4,5}                      
		\\
		& массив        &          &            &             &
		\\ \hline
		\multirow{2}{*}{2}& \multirow{2}{*}{Отсортированный} & \multirow{2}{*}{5,4,3,2,1} & \multirow{2}{*}{1,2,3,4,5} & \multirow{2}{*}{1,2,3,4,5}   &  \multirow{2}{*}{1,2,3,4,5}                      
		\\
		& в обратном порядке        &          &            &             &
		\\ \hline
		\multirow{2}{*}{2}& \multirow{2}{*}{Случайные} & \multirow{2}{*}{-8,3,1,6,-2} & \multirow{2}{*}{-8,-2,1,3,6} & \multirow{2}{*}{-8,-2,1,3,6}   &  \multirow{2}{*}{-8,-2,1,3,6}                      
		\\
		& числа        &          &            &             &
		\\ \hline
	\end{tabular}
\end{table}

\section*{Вывод}
В данном разделе был обоснован выбор языка программирования. Реализованы функции сортировки <<блинная>>, <<быстрая>>, <<выбором>>, и проведено их тестирование методом черного ящика по таблице \ref{table:ref2}. 