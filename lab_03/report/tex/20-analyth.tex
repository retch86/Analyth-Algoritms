\chapter{Аналитический раздел}
В разделе приводятся критерии выбора алгоритма сортировки, а также описание алгоритмов сортировок <<блинная>>, <<быстрая>>, <<выбором>>.
\section{Понятие сортировки}
Сортировка — процесс упорядочения элементов в списке по возрастанию или убыванию. В случае, когда элемент списка имеет несколько полей, поле, служащее критерием порядка, называется ключом сортировки. На практике в качестве ключа часто выступает число, а в остальных полях хранятся какие-либо данные, никак не влияющие на работу алгоритма.

\section{Критерии выбора алгоритма сортировки}
К основным параметрам выбора необходимого алгоритма сортировки относят:
\begin{itemize}
	\item временная сложность: описывает то, как производительность алгоритма изменяется в зависимости от размера набора данных;
	\item память: ряд алгоритмов требует выделения дополнительной памяти под временное хранение данных. При оценке не учитывается место, которое занимает исходный массив и независящие от входной последовательности затраты, например, на хранение кода программы;
	\item устойчивость: сортировка является устойчивой в том случае, если для любой пары элементов с одинаковым ключами, она не меняет их порядок в отсортированном списке (является важным критерием для баз данных).
\end{itemize} 

\section{Блинная сортировка}
Блинная сортировка получила свое название от аналогии с переворачиванием блинов на сковороде \cite{sort_pancake}. Метод относится к классу сортировки обмена элементов путем их сравнения, как в пузырьковой. Сначала необходимо найти максимальный элемент в массиве. Далее переворачиваем (выполняем обмен) элементом от левого края до максимального -- в результате максимум окажется на левом крае. На последнем шаге переворачивается весь неотсортированный подмассив, тогда максимальный элемент попадет на свое место. Все эти действия повторяются с оставшейся, неотсортированной, частью массива. 

\section{Быстрая сортировка}
В алгоритме быстрой сортировки может использоваться как рекурсивный, так и итеративный подход. Рассмотрим первый случай. Выбрав опорный элемент в списке, данный алгоритм сортировки делит массив на две части, относительно выбранного элемента \cite{quick_sort}. Далее в первую часть попадают все элементы, меньшие выбранного, а во вторую — б\'ольшие. Если в данных частях более двух элементов, рекурсивно запускается для него та же процедура. В конце получится полностью отсортированная последовательность.

\section{Сортировка выбором}
Алгоритм сортировки выбором совершает несколько проходов по списку. При каждом проходе выбирается минимальный из еще не отсортированных элементов и обменивается с первым элементом не отсортированной области \cite{sort_select}. В следующем проходе рассмотренный элемент не участвует, сортируется только оставшийся хвост.

Для реализации устойчивости алгоритма необходимо минимальный элемент непосредственно вставлять в первую не отсортированную позицию, не меняя порядок остальных элементов.

\section*{Вывод}
В данном разделе были рассмотрены основные теоретические сведения об алгоритмах сортировки <<блинная>>, <<быстрая>>, <<выбором>>. 