\chapter{Аналитический раздел}
В данном разделе рассматривается базовый алгоритм сортировки слиянием, а также идея его параллельной версии. 
\section{Сортировка слиянием}
\subsection{Последовательная версия}
Сортировка слиянием определяется как алгоритм сортировки, который работает путем разделения массива на более мелкие подмассивы, сортировки каждого подмассива, а затем обратного слияния отсортированных подмассивов для формирования окончательного отсортированного массива \cite{seq_sort}.

Пример сортировки представлен на рисунке \ref{img:merge_sort_ex}.
\imgw{merge_sort_ex.pdf}{Пример сортировки слиянием}{img:merge_sort_ex}{0.65\textwidth}

\subsection{Параллельная версия}
Идея параллельного выполнения сортировки слиянием представлена на рисунке \ref{img:merge_sort_parall_ex}.
\imgw{merge_sort_parall_ex.pdf}{Идея параллельного выполнения сортировки слиянием}{img:merge_sort_parall_ex}{0.65\textwidth}
Пусть есть два потока thread 0 и thread 1. Разделим исходный массив на две части таким образом, что одна часть будет обработана потоком thread 0, а другая -- потоком thread 1 \cite{parall_sort}. После этого каждый из потоков будет рекурсивно применять алгоритм сортировки слиянием к своей части массива. На выходе из этого шага получится два отсортированных локально фрагмента массива. Затем эти фрагменты объединяются для формирования глобально отсортированного массива. На нижней части рисунка \ref{img:merge_sort_parall_ex} показано, что для получения глобально отсортированного массива требуется выполнить рекурсивное слияние.

\section{Использование средств синхронизации}
В таблице \ref{tbl:synchro} приведена сравнительная таблица необходимости использования средств синхронизации.
\begin{table}[H]
	\centering
	\captionsetup{justification=raggedleft}
	\caption{Сравнение средств синхронизации}
	\label{tbl:synchro}
	\begin{tabular}{|c|l|c|}
		\hline
		\textbf{\begin{tabular}[c]{@{}c@{}}Средство\\ синхронизации\end{tabular}} & \multicolumn{1}{c|}{\textbf{Описание}}                                                                                                                                                                                      & \textbf{\begin{tabular}[c]{@{}c@{}}Необхо-\\ димость\\ использо-\\ вания\end{tabular}} \\ \hline
		Мьютекс                                                                   & \begin{tabular}[c]{@{}l@{}}Переменная, блокирующая доступ\\ к разделяемым ресурсам. \\ Только поток-владелец\\ мьютекса, захвативший его, может его\\ освободить.\end{tabular}                                              & -                                                                                      \\ \hline
		Семафоры                                                                  & \begin{tabular}[c]{@{}l@{}}Неотрицательная защищенная переменнная,\\ на которой определены операции: \\ захватить P и освободить V. Семафор может \\ захватить один процесс, а\\ освободить совершенно другой.\end{tabular} & -                                                                                      \\ \hline
	\end{tabular}
\end{table}

Использование таких средств синхронизации как мьютекс или семафор не требуется для распараллеливания алгоритма сортировки слиянием. Требуется лишь ожидание завершение всех выделенных потоков процесса кроме главного, а после выполнить операцию слияния результата.

\section*{Вывод}
В разделе была рассмотрена последовательная версия сортировки слиянием, предложена параллельная версия за счет разделения исходного массива на части, которые будут обрабатывать отдельные потоки. В качестве средства синхронизации выбрано ожидание завершения всех потоков процесса кроме главного.