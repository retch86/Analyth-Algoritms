\part*{ЗАКЛЮЧЕНИЕ}
\addcontentsline{toc}{part}{\textbf{ЗАКЛЮЧЕНИЕ}}

В ходе выполнения лабораторной работы были рассмотрены последовательная и параллельная реализация алгоритма сортировки слиянием. Были выполнено описание каждого из этих алгоритмов, приведены соответствующие схемы алгоритмов. 

Многопоточная версия алгоритма демонстрирует ускорение в среднем в 2.5 раза по сравнению с последовательной реализацией. Однако использование одного дополнительного потока не является эффективным решением в качестве замены последовательной версии из-за затрат на его создание, выделение ресурсов и уничтожение, что приводит к приросту времени выполнения в среднем в 1.5-2 раза. Оптимальным выбором оказывается использование 8 потоков, что минимизирует время работы алгоритма сортировки слиянием по сравнению с другими вариантами. Для наилучшего распараллеливания рекомендуется использовать количество потоков, соответствующее числу логических ядер устройства.