\part*{ВВЕДЕНИЕ}
\addcontentsline{toc}{part}{\textbf{ВВЕДЕНИЕ}}

Параллельные вычисления позволяют увеличить скорость выполнения программ. Конвейерная обработка данных является популярным приемом при работе с параллельностью. Она позволяет на каждой следующей «линии» конвейера использовать данные, полученные с предыдущего этапа.

Конвейер — способ организации вычислений, используемый в современных процессорах и контроллерах с целью повышения их производительности.

Целью данной лабораторной работы является изучение принципов конвейерной обработки данных.

Для достижения поставленной цели необходимо выполнить следующие задачи:

\begin{itemize}[label=---]
	\item исследовать основы конвейерной обработки данных;
	\item привести схемы алгоритмов, используемых для конвейерной и линейной обработок данных;
	\item определить средства программной реализации;
	\item провести модульное тестирование;
	\item провести сравнительный анализ времени работы алгоритмов;
	\item описать и обосновать полученные результаты.
\end{itemize}