\usepackage[english,main=russian]{babel}

%\usepackage{newtxmath}
\usepackage[no-math]{fontspec}
\usepackage{polyglossia}

\defaultfontfeatures{Ligatures = TeX, Mapping = tex-text}

\setmainlanguage[babelshorthands = true]{russian}
\setotherlanguage{english}

\setmainfont{Times New Roman}

\newfontfamily\cyrillicfont{Times New Roman}
\newfontfamily\englishfont{Times New Roman}

\usepackage[
left=30mm,
right=10mm, 
top=20mm,
bottom=20mm,
]{geometry}

\makeatletter
\renewcommand\LARGE{\@setfontsize\LARGE{22pt}{20}}
\renewcommand\Large{\@setfontsize\Large{20pt}{20}}
\renewcommand\large{\@setfontsize\large{16pt}{20}}
\makeatother

\usepackage{microtype} % Настройка переносов
\sloppy

\usepackage{setspace} % Настройка межстрочного интервала
\onehalfspacing

\usepackage{indentfirst} % Настройка абзацного отступа
\setlength{\parindent}{12.5mm}

\usepackage[unicode,hidelinks]{hyperref}
\usepackage{xifthen}

\usepackage{colortbl}

\usepackage[normalem]{ulem}
% Текст под линией 
\newcommand*{\undertext}[2]{%
	\begin{tabular}[t]{@{}c@{}}%
		#1\\\relax\scriptsize(#2)%
	\end{tabular}
}


\newcolumntype{T}{>{\centering\arraybackslash}p{0.3\textwidth}}
\newcolumntype{P}{>{\centering\arraybackslash}p{0.2\textwidth}}

\usepackage[figure,table]{totalcount} % Подсчет изображений, таблиц
\usepackage{rotating} % Поворот изображения вместе с названием
\usepackage{lastpage} % Для подсчета числа страниц

\usepackage{titlesec}
\usepackage{titletoc}
\usepackage{tocloft}

\setcounter{tocdepth}{5}

\setlength{\cftbeforetoctitleskip}{-25pt}
\renewcommand{\cfttoctitlefont}{\large\bfseries}

\renewcommand{\cftchapfont}{\large\bfseries}
\renewcommand{\cftsecfont}{\large}
\renewcommand{\cftchapleader}{\cftdotfill{\cftdotsep}}

\setlength{\cftbeforepartskip}{10pt}


\setcounter{secnumdepth}{5}

\titleformat{\part}[block]
{\large\bfseries}{\thechapter}{0.5em}{\large\centering}

\titleformat{\partappendix}
{\large\bfseries}{\thechapter}{0.5em}{\large\centering}

\titleformat{\chapter}[block]
{\large\bfseries}{\thechapter}{0.5em}{\large\raggedright}

\titleformat{\section}[block]
{\large\bfseries}{\thesection}{0.5em}{\large\raggedright}
\renewcommand{\thesection}{\arabic{chapter}.\arabic{section}}

\titleformat{\subsection}[block]
{\hspace{\parindent}\large\bfseries}{\thesubsection}{0.5em}{\large\raggedright}
\renewcommand{\thesubsection}{\arabic{chapter}.\arabic{section}.\arabic{subsection}}

\titleformat{\subsubsection}[block]
{\hspace{\parindent}\large\bfseries}{\thesubsubsection}{0.5em}{\large\raggedright}
\renewcommand{\thesubsubsection}{\arabic{chapter}.\arabic{section}.\arabic{subsection}.\arabic{subsubsection}}

\titleclass{\part}{top}
\titlespacing*{\part}{12.5mm}{-22pt}{10pt}

\titlespacing{\chapter}{12.5mm}{-22pt}{10pt}
\titlespacing{\section}{12.5mm}{10pt}{10pt}
\titlespacing{\subsection}{12.5mm}{10pt}{10pt}
\titlespacing{\subsubsection}{12.5mm}{10pt}{10pt}

% ---------------------------------------- CAPTION --------------------------------- %

\usepackage[
labelsep=endash,
singlelinecheck=false,
]{caption}

\captionsetup[figure]{justification=centering}
\captionsetup[table]{justification=raggedleft}
\captionsetup[listing]{justification=raggedright}


% ---------------------------------------- ABBRS --------------------------------- %

\usepackage{enumitem}
\newcounter{descriptcount}
\newlist{enumdescript}{description}{2}
\setlist[enumdescript,1]{%
	before={\setcounter{descriptcount}{0}%
		\renewcommand*\thedescriptcount{\arabic{descriptcount})}}
	,font=\stepcounter{descriptcount}\thedescriptcount~
}
\setlist[enumdescript,2]{%
	before={\setcounter{descriptcount}{0}%
		\renewcommand*\thedescriptcount{\roman{descriptcount}}}
	,font=\stepcounter{descriptcount}\thedescriptcount~
}

\def\labelitemi{--} % Изменение буллета для списков


% ---------------------------------------- TABLE  ----------------------------------------

\usepackage{xcolor}
\usepackage{tabularx}
\usepackage{booktabs}
\usepackage{multirow}
\usepackage{longtable}

% ---------------------------------------- FIGURE ----------------------------------------

\usepackage{graphicx}
\usepackage{float}
\usepackage{wrapfig}
\usepackage{tikzscale}
\usepackage[notransparent]{svg}

\usepackage{pgfplots}
\pgfplotsset{compat=newest}

% ----------------------------------------- MATH -----------------------------------------

\usepackage{lscape}
\usepackage{afterpage}

\usepackage{amsmath}
% ----------------------------------------- LST -----------------------------------------
\usepackage{listings}
\usepackage{courier}

%\usepackage[cache=false, newfloat]{minted}
%\newenvironment{code}{\captionsetup{type=listing}}{}
\renewcommand{\lstlistingname}{Листинг}
%\usepackage{listings-golang} % import this package after listings
\newcommand{\commentfont}{\fontfamily{pcr}}


%\usepackage{minted}
\lstset{
	basicstyle=\fontfamily{pcr}\footnotesize,
	numbers=left,
	numberstyle=\fontfamily{pcr}\tiny\color{black},
	numbersep=10pt,
	tabsize=4,
	keywordstyle=\fontfamily{pcr}\bfseries\color{black},
	frame=tblr,	
	xleftmargin=20pt,
	framexleftmargin=20pt,
	framexrightmargin=0pt,
	framexbottommargin=5pt,
	language=C++,
	inputencoding=utf8,
	extendedchars=true,
	keepspaces=true,
}

% ----------------------------------------- BIBLIO ---------------------------------------

\usepackage{totcount}
%\newtotcounter{citenum} %From the package documentation
%\AtEveryBibitem{\stepcounter{citenum}}

% пакеты для псевдокода
\usepackage{algorithm}
\usepackage{algpseudocode}
\floatname{algorithm}{Алгоритм}
\captionsetup[ruled]{labelsep=period}
\makeatletter
\@addtoreset{algorithm}{chapter}% algorithm counter resets every chapter
\makeatother
\renewcommand{\thealgorithm}{\thechapter.\arabic{algorithm}}%

%Перевод команд псевдокода
\algrenewcommand\algorithmicwhile{\textbf{До тех пока}}
\algrenewcommand\algorithmicdo{\textbf{выполнять}}
\algrenewcommand\algorithmicrepeat{\textbf{Повторять}}
\algrenewcommand\algorithmicuntil{\textbf{Пока выполняется}}
\algrenewcommand\algorithmicend{\textbf{Конец}}
\algrenewcommand\algorithmicif{\textbf{Если}}
\algrenewcommand\algorithmicelse{\textbf{иначе}}
\algrenewcommand\algorithmicthen{\textbf{тогда}}
\algrenewcommand\algorithmicfor{\textbf{Цикл}}
\algrenewcommand\algorithmicforall{\textbf{Выполнить для всех}}
\algrenewcommand\algorithmicfunction{\textbf{Функция}}
\algrenewcommand\algorithmicprocedure{\textbf{Процедура}}
\algrenewcommand\algorithmicloop{\textbf{Зациклить}}
\algrenewcommand\algorithmicrequire{\textbf{Условия:}}
\algrenewcommand\algorithmicensure{\textbf{Обеспечивающие условия:}}
\algrenewcommand\algorithmicreturn{\textbf{Вернуть}}
\algrenewtext{EndWhile}{\textbf{Конец цикла}}
\algrenewtext{EndLoop}{\textbf{Конец зацикливания}}
\algrenewtext{EndFor}{\textbf{Конец цикла}}
\algrenewtext{EndFunction}{\textbf{Конец функции}}
\algrenewtext{EndProcedure}{\textbf{Конец процедуры}}
\algrenewtext{EndIf}{\textbf{Конец условия}}
\algrenewtext{EndFor}{\textbf{Конец цикла}}
\algrenewtext{BeginAlgorithm}{\textbf{Начало алгоритма}}
\algrenewtext{EndAlgorithm}{\textbf{Конец алгоритма}}
\algrenewtext{BeginBlock}{\textbf{Начало блока. }}
\algrenewtext{EndBlock}{\textbf{Конец блока}}
\algrenewtext{ElseIf}{\textbf{иначе если }}

\usepackage{amssymb}

%-----------------------------------------TABLE---------------------------------------------
\usepackage{tabularx}
\newcolumntype{Y}{>{\centering\arraybackslash}X}

%---------------------------------------REFERENCES------------------------------------------
\newtotcounter{citnum} %From the package documentation
\def\oldbibitem{} \let\oldbibitem=\bibitem
\def\bibitem{\stepcounter{citnum}\oldbibitem}

\usepackage{pdfpages}

\captionsetup[figure]{name={Рисунок}, labelsep=endash}

%-------------------------------------------IMAGES------------------------------------------
%Настройка команды вставки изображения
\newcommand{\imgs}[4]
{
	\begin{figure}[H]
		\captionsetup{justification=centering}
		\centering{
			\includegraphics[scale=#4]{images/#1}
			\caption{#2}
			\label{#3}
		}
	\end{figure}
}

\newcommand{\imgw}[4]
{
	\begin{figure}[H]
		\captionsetup{justification=centering}
		\centering{
			\includegraphics[width=#4]{images/#1}
			\caption{#2}
			\label{#3}
		}
	\end{figure}
}

%---------------------------------------------TITLE-------------------------------------------
% Установка мета-данных выходного файла
\newcommand{\documentmeta}[4]
{
	\hypersetup{
		pdftitle={#1 #2 #3},
		pdfsubject={#4},
		pdfauthor={#2}
	}
}

% Cоздание полей
\RequirePackage[normalem]{ulem}
\RequirePackage{stackengine}
\newcommand{\longunderline}[1]
{
	#1\hfill\mbox{}
}
\newcommand{\fixunderline}[3]
{
	$\underset{\text{#3}}{\text{\uline{\stackengine{0pt}{\hspace{#2}}{\text{#1}}{O}{c}{F}{F}{L}}}}$
}

% Создание горизонтальной линии
\makeatletter
\newcommand{\vhrulefill}[1]
{
	\leavevmode\leaders\hrule\@height#1\hfill \kern\z@
}
\makeatother

% Создание шапки титульной страницы
\newcommand{\titlepageheader}[2]
{
	\begin{wrapfigure}[7]{l}{0.14\linewidth}
		\vspace{3.4mm}
		\hspace{-8mm}
		\includegraphics[width=0.89\linewidth]{bmstu-logo}
	\end{wrapfigure}
	
	{
		\singlespacing \small
		\textbf{Министерство науки и высшего образования Российской Федерации \\
		Федеральное государственное бюджетное образовательное учреждение \\
		высшего образования \\
		<<Московский государственный технический университет \\
		имени Н.~Э.~Баумана \\
		(национальный исследовательский университет)>> \\
		(МГТУ им. Н.~Э.~Баумана) \\}
	}
	
	\vspace{-4.2mm}
	\vhrulefill{0.9mm} \\
	\vspace{-7mm}
	\vhrulefill{0.2mm} \\
	\vspace{2.8mm}
	
	{
		\small
		ФАКУЛЬТЕТ ИУ\longunderline{<<#1>>} \\
		\vspace{3.3mm}
		КАФЕДРА ИУ-7\longunderline{<<#2>>} \\
	}
}

% Установка заголовков отчета по НИР
\newcommand{\titlepageresearchtitle}[1]
{
	{
		\LARGE \bfseries
		РАСЧЕТНО-ПОЯСНИТЕЛЬНАЯ ЗАПИСКА \\
	}
	\vspace{5mm}
	{
		\Large \itshape
		К НАУЧНО-ИССЛЕДОВАТЕЛЬСКОЙ РАБОТЕ \\
		\vspace{5mm}
		НА ТЕМУ: \\
		<<#1>> \\
	}
}

% Установка заголовков РПЗ
\newcommand{\titlepagenotetitle}[2]
{
	{
		\LARGE \bfseries
		РАСЧЕТНО-ПОЯСНИТЕЛЬНАЯ ЗАПИСКА \\
	}
	\vspace{5mm}
	{
		\Large \itshape
		#1 \\
		\vspace{5mm}
		НА ТЕМУ: \\
		<<#2>> \\
	}
}

% Установка заголовков отчета
\newcommand{\titlepagereporttitle}[4]
{
	\vspace{2.5cm}
	\textbf{\Large ОТЧЕТ} \\
	\large по #1 \\
	\ifthenelse{\isempty{#2}}{}{по курсу <<#2>> \\}
	\ifthenelse{\isempty{#3}}{}{на тему: <<#3>> \\}
	\ifthenelse{\isempty{#4}}{}{Вариант №~#4 \\}
}

\newcommand{\signfontsize}
{
	\fontsize{14pt}{2cm}\selectfont
}

% Создание поля студента
\RequirePackage{pgffor}

\newcommand*\titlepagestudentscontent{}

\newcommand{\maketitlepagestudent}[3]
{
	%\foreach \s/\g in {#1} {
	%	\gappto\titlepagestudentscontent{\signfontsize Студент \hspace{2.5cm}\fixunderline}
	%	\xappto\titlepagestudentscontent{{\g}}
	%	\gappto\titlepagestudentscontent{{25mm}{(Группа)} &}
	%	\gappto\titlepagestudentscontent{\fixunderline{}{40mm}{(Подпись, дата)} \vspace{1.3mm} &}
	%	\gappto\titlepagestudentscontent{\fixunderline}
	%	\xappto\titlepagestudentscontent{{\noexpand\signfontsize\s}}
	%	\gappto\titlepagestudentscontent{{40mm}{(И.~О.~Фамилия)} \\}
	%}
	
	\foreach \c in {#3} {
		\gappto\titlepagestudentscontent{\signfontsize#1 \hspace{2.5cm} #2 &}
		\gappto\titlepagestudentscontent{\fixunderline{}{40mm}{(Подпись, дата)} \vspace{1.3mm} &}
		%\gappto\titlepagestudentscontent{\fixunderline}
		\xappto\titlepagestudentscontent{{\noexpand\signfontsize\c} \\}
		%\gappto\titlepagestudentscontent{{40mm}{(И.~О.~Фамилия)} \\}
	}
}

% Создание прочих полей
\newcommand*\titlepageotherscontent{}

\newcommand{\maketitlepageothers}[2]
{
	\foreach \c in {#2} {
		\gappto\titlepagestudentscontent{\signfontsize#1 &}
		\gappto\titlepagestudentscontent{\fixunderline{}{40mm}{(Подпись, дата)} \vspace{1.3mm} &}
		%\gappto\titlepagestudentscontent{\fixunderline}
		\xappto\titlepagestudentscontent{{\noexpand\signfontsize\c}}
		%\gappto\titlepagestudentscontent{{40mm}{(И.~О.~Фамилия)} \\}
	}
}

% Установка исполнителей работы
\newcommand{\titlepageauthors}[7]
{
	{
		\renewcommand{\titlepagestudentscontent}{}
		\maketitlepagestudent{Студент}{#1}{#2}
		
		\renewcommand{\titlepageotherscontent}{}
		\maketitlepageothers{#3}{#4}
		\maketitlepageothers{Консультант}{#5}
		\maketitlepageothers{Нормоконтролер}{#6}
		
		\small
		\begin{tabularx}{\textwidth}{@{}>{\hsize=.5\hsize}X>{\hsize=.25\hsize}X>{\hsize=.25\hsize}X@{}}
			\titlepagestudentscontent
			
			\titlepageotherscontent
		\end{tabularx}
	}
}

% Создание титульной страницы РПЗ к ВКР
\newcommand{\makethesistitle}[7]
{
	\documentmeta{РПЗ к ВКР}{#4}{}{#3}
	
	\begin{titlepage}
		\centering
		
		\titlepageheader{#1}{#2}
		\vspace{15.8mm}
		
		\titlepagenotetitle{К ВЫПУСКНОЙ КВАЛИФИКАЦИОННОЙ РАБОТЕ}{#3}
		\vfill
		
		\titlepageauthors{#4}{Руководитель ВКР}{#5}{#6}{#7}
		\vspace{14mm}
		
		\textit{{\the\year} г.}
	\end{titlepage}

}

% Создание титульной страницы отчета по НИР
\newcommand{\makeresearchtitle}[6]
{
	\documentmeta{Отчет по НИР}{#4}{}{#3}
	
	\begin{titlepage}
		\centering
		
		\titlepageheader{#1}{#2}
		\vspace{15.8mm}
		
		\titlepageresearchtitle{#3}
		\vfill
		
		\titlepageauthors{#4}{Руководитель НИР}{#5}{#6}{}
		\vspace{14mm}
		
		\textit{{\the\year} г.}
	\end{titlepage}
	
}

% Создание титульной страницы РПЗ к КР
\newcommand{\makecourseworktitle}[6]
{
	\documentmeta{РПЗ к КР}{#4}{}{#3}
	
	\begin{titlepage}
		\centering
		
		\titlepageheader{#1}{#2}
		\vspace{15.8mm}
		
		\titlepagenotetitle{К КУРСОВОЙ РАБОТЕ}{#3}
		\vfill
		
		\titlepageauthors{#4}{Руководитель курсовой работы}{#5}{#6}{}
		\vspace{14mm}
		
		\textit{{\the\year} г.}
	\end{titlepage}
	
}

% Создание титульной страницы отчета
\newcommand{\makereporttitle}[9]
{
	\documentmeta{Отчет}{#7}{по #3 по курсу #4}{#5}
	
	\begin{titlepage}
		\centering
		
		\titlepageheader{#1}{#2}
		\vspace{15.8mm}
		
		\titlepagereporttitle{#3}{#4}{#5}{#6}
		\vfill
		
		\titlepageauthors{#7}{#8}{Преподаватель}{#9}{}{}{}
		\vspace{14mm}
		
		\textit{{\the\year} г.}
	\end{titlepage}
	
}