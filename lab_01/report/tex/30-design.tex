\chapter{Конструкторский раздел}
В данном разделе приводятся схемы итеративного и рекурсивного алгоритмов расстояния Левенштейна, приводится структура программы.

\section{Алгоритмы Левенштейна}
\subsection{Итеративный алгоритм Левенштейна}
На рисунке \ref{img:iter_levenstein} представлен итеративный алгоритм Левенштейна с двумя строками. 
\imgw{iter_levenstein.pdf}{Итеративный алгоритм Левенштейна с двумя строками}{img:iter_levenstein}{0.95\textwidth}

\subsection{Рекурсивный алгоритм Левенштейна без кэша}
На рисунке \ref{img:recur_levenstein_no_cache} представлен рекурсивный алгоритм Левенштейна без кэша. 
\imgw{recur_levenstein_no_cache.pdf}{Рекурсивный алгоритм Левенштейна без кэша}{img:recur_levenstein_no_cache}{0.95\textwidth}

\subsection{Рекурсивный алгоритм Левенштейна с матрицей}
На рисунке \ref{img:recur_levenstein_mtrx_1} и \ref{img:recur_levenstein_mtrx_2} представлен рекурсивный алгоритм Левенштейна с матрицей. 
\imgw{recur_levenstein_mtrx_1.pdf}{Рекурсивный алгоритм Левенштейна с матрицей}{img:recur_levenstein_mtrx_1}{0.95\textwidth}
\imgw{recur_levenstein_mtrx_2.pdf}{Рекурсивный алгоритм Левенштейна с матрицей (продолжение)}{img:recur_levenstein_mtrx_2}{0.75\textwidth}

\subsection{Рекурсивный алгоритм Дамерау-Левенштейна}
На рисунке \ref{img:recur_damerau_levenstein} представлен рекурсивный алгоритм Дамерау-Левенштейна. 
\imgw{recur_damerau_levenstein.pdf}{Рекурсивный алгоритм Дамерау-Левенштейна}{img:recur_damerau_levenstein}{0.95\textwidth}

\section{Структура программы}
Программа состоит из следующих модулей:
\begin{itemize}
	\item main.py: основной файл программы, в котором вызываются алгоритмы нахождения расстояния Левенштейна и Дамерау-Левенштейна;
	\item distance.py: файл, содержащий код всех представленных алгоритмов;
	\item test\_time.py: замер времени выполнения каждого из алгоритмов;
	\item generate\_string.py: генерация строки заданного размера;
	\item graph.py: файл отображение результатов замеров зависимости времени работы алгоритмов от входной строки;
	\item menu.py: перечень команд для взаимодействия с программой.
\end{itemize}

\section*{Вывод}
В данном разделе были приведены схемы итеративного и рекурсивного алгоритма Левенштейна и рекурсивного Дамерау-Левенштейна, а также расписана структура программы.
