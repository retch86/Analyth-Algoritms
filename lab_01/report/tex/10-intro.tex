\part*{ВВЕДЕНИЕ}
\addcontentsline{toc}{part}{\textbf{ВВЕДЕНИЕ}}

При наборе текста выявляются трудности из-за опечаток. Возникает необходимость в эффективных средствах для их быстрого исправления.

Для решения подобных проблем в области прикладной лингвистики существует направление, известное как компьютерная лингвистика. В ней разрабатываются и применяются компьютерные программы, предназначенные для исследования языка и моделирования его функционирования в различных условиях \cite{big_comp_lingua}.

Одним из первых , был советский ученый В. И. Левенштейн \cite{levenstein}. Его алгоритм стал известен как расстояние Левенштейна — метрика, измеряющая различие между двумя строками в количестве редакторских операций (вставки, удаления, замены), необходимых для преобразования одной последовательности символов в другую. Расстояние Дамерау-Левенштейна является модификацией этого алгоритма, добавляя к редакторским операциям еще и транспозицию, обмен двух соседних символов. Эти алгоритмы нашли применение не только в компьютерной лингвистике, но также в биоинформатике для оценки схожести различных участков ДНК и РНК.

Целью данной лабораторной работы является изучение, реализация и исследование алгоритмов нахождения расстояний Левенштейна и \text{Дамерау-Левенштейна}. Для достижения поставленной цели необходимо решить следующий набор задач:

\begin{itemize}
	\item проанализировать алгоритмы Левенштейна и Дамерау-Левештейна;
	\item формально описать алгоритмы Левенштейна и Дамерау-Левештейна;
	\item выполнить тестирование реализации алгоритмов методом черного ящика;
	\item определить зависимость времени выполнения и необходимой памяти для функционирования предлагаемой реализации от размерности входных данных;
	\item привести рекомендации по использованию алгоритмов.
\end{itemize}