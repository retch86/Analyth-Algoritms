\renewcommand\bibname{\centerline{СПИСОК ИСПОЛЬЗОВАННЫХ ИСТОЧНИКОВ}}
\addcontentsline{toc}{part}{\textbf{СПИСОК ИСПОЛЬЗОВАННЫХ ИСТОЧНИКОВ}}

\begin{thebibliography}{3}
	\makeatletter
	\def\@biblabel#1{#1. }
	
	\bibitem{big_comp_lingua}
	КОМПЬЮТЕРНАЯ ЛИНГВИСТИКА Большая российская энциклопедия - электронная версия [Электронный \text{ресурс]. -- Режим} доступа, URL: \urlstyle{same}\url{https://bigenc.ru/linguistics/text/2087783} (дата обращения: 05.02.2024)
	
	\bibitem{levenstein}
	Левенштейн В. И. Двоичные коды с исправлением выпадений, вставок и замещений символов //Доклады Академии наук. – Российская академия наук, 1965. – Т. 163. – №. 4. – С. 845-848.
	
	\bibitem{damerau_levenshtein}
	Damerau F. J. A technique for computer detection and correction of spelling errors //Communications of the ACM. – 1964. – Т. 7. – №. 3. – С. 171-176.
	
	\bibitem{python}
	Python 3.12.1 documentation [Электронный \text{ресурс]. -- Режим} доступа, URL: \urlstyle{same}\url{https://docs.python.org/3/} (дата обращения: 05.02.2024)

	\bibitem{process_time}
	time — Time access and conversions — Python 3.12.1 documentation [Электронный \text{ресурс]. -- Режим} доступа, URL: \urlstyle{same}\url{https://docs.python.org/3/library/time.html} (дата обращения: 05.02.2024)
	
	\bibitem{profiler}
	The Python Profilers — Python 3.12.1 documentation [Электронный \text{ресурс]. -- Режим} доступа, URL: \urlstyle{same}\url{https://docs.python.org/3/library/profile.html} (дата обращения: 05.02.2024)
\end{thebibliography}