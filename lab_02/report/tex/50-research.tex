\chapter{Исследовательский раздел}
В данном разделе приводятся технические характеристики устройства, анализ зависимости времени выполнения алгоритмов от четного и нечетного размера квадратных матриц.
\section{Технические характеристики}
Технические характеристии устройства, на котором выполнялось тестирование:
\begin{itemize}
	\item операционная система: Windows 10 Pro;
	\item память: 8 GiB;
	\item процессор: Intel(R) Core(TM) i5-8265U CPU @ 1.60GHz   1.80 GHz.
\end{itemize}
Тестирование проводилось на ноутбуке, который был подключен к сети питания. Во время проведения тестирования ноутбук был нагружен только встроенными приложениями окружения, самим окружением и системой тестирования.

\section{Временные характеристики выполнения}
Проведем анализ зависимости времени работы алгоритмов умножения матриц от размера исходных матриц. Рассмотрим вариант для лучшего случая, когда размер матрицы имеет четные значения, и для худшего при нечетном размере матриц. Исходными данными является квадратная матрица целых чисел. Единичные замеры выдадут крайне маленький результат, поэтому проведем работу каждого алгоритма n = 5 раз и поделим на число n. Получим среднее значение работы каждого из алгоритмов. 

Выполним анализ для случая, когда размер квадратных матриц целых чисел имеет четное значение \{100, 200,..., 1000\}. Результат зависимости времени выполнения умножения матриц четного размера приведен на рисунке \ref{fg:6_1}.

\begin{figure}[H]
	\centering
	\begin{tikzpicture}
		\begin{axis}
			[grid = major,
			xlabel = Размер матрицы,
			ylabel = {Время, c},
			ymin = 0,
			width = 0.95\textwidth,
			height=0.3\textheight,
			legend style={at={(0.5,-0.3)},anchor=north},
			xmajorgrids=true]
			\addplot table{data/even_standart.txt};
			\addplot table{data/even_vinograd.txt};
			\addplot table{data/even_vinograd_optimized.txt};
			\addplot table{data/even_strassen.txt};
			\legend{
				Стандартный алгоритм,
				Алгоритм Винограда,
				Оптимизированный алгоритм Винограда,
				Алгоритм Штрассена
			};
		\end{axis}
	\end{tikzpicture}
	\caption{Зависимости времени работы алгоритмов при чётных размерностях матриц} 
	\label{fg:6_1}
\end{figure}

Алгоритм умножения матриц по Винограду работает медленнее стандартного приблизительно на 9\%. Оптимизированная версия алгоритма Винограда выполняет вычисления быстрее стандартного на размерах \{700, ..., 1000\} в среднем на 40\% и выигрывает по скорости у обычного алгоритма умножения по Винограду на 51\% на размерах \{700, ..., 1000\}. Алгоритм Штрассена медленнее других способов умножения матриц в среднем в 15-17 раз до размера матриц 500; с 500 выполняется резкое увеличение времени выполнения алгоритма Штрассена. Разница с другими алгоритмами составляет в среднем 8-10 раз.


Выполним анализ для случая, когда размер матриц целых чисел имеет нечетный размер \{101, 201,..., 1001\}. Результат приведен на рисунке \ref{fg:6_2}.
\begin{figure}[H]
	\centering
	\begin{tikzpicture}
		\begin{axis}
			[grid = major,
			xlabel = Размер матрицы,
			ylabel = {Время, c},
			ymin = 0,
			width = 0.95\textwidth,
			height=0.3\textheight,
			legend style={at={(0.5,-0.3)},anchor=north},
			xmajorgrids=true]
			
			\addplot table{data/odd_standart.txt};
			\addplot table{data/odd_vinograd.txt};
			\addplot table{data/odd_vinograd_optimized.txt};
			\addplot table{data/odd_strassen.txt};
			\legend{
				Стандартный алгоритм,
				Алгоритм Винограда,
				Оптимизированный алгоритм Винограда,
				Алгоритм Штрассена
			};
		\end{axis}
	\end{tikzpicture}
	\caption{График зависимости времени работы алгоритмов при нечётных размерностях матриц} 
	\label{fg:6_2}
\end{figure}
Алгоритм умножения матриц по Винограду работает медленнее стандартного приблизительно на 12\% на размерах \{701, ..., 1001\}. Оптимизированная версия алгоритма Винограда выполняет вычисления быстрее стандартного примерно на 47\% и выигрывает по скорости у обычного алгоритма умножения по Винограду в среднем на 63\% на размерах \{701, ..., 1001\}.


\section*{Вывод}
Экспериментально была подтвержена трудоемкость алгоритмов умножения матриц, описанная в разделах 2.3.1, 2.3.2, 2.3.3, 2.3.4. Время выполнения, описанное в разделе 4.2., каждого из этих методов при нечетных размерностях матриц больше времения выполнения тех же методов в случае, когда размерность матриц четная. Это связано с дополнительными операциями обработки, указанных в схемах алгоритма раздела \ref{section:shemas_algo}.