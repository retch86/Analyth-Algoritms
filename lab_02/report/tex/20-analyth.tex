\chapter{Аналитический раздел}
В данном разделе приводится анализ алгоритмов умножения матриц, а также метод оптимизации вычислений.
\section{Определение матрицы}
Матрицей размера $m \times n$ называется прямоугольная таблица элементов некоторого множества 
(например, чисел или функций), имеющая $m$ строк и $n$ столбцов \cite{angem}.
Элементы $a_{ij}$ , из которых составлена матрица, называются элементами матрицы.
Условимся, что первый индекс $i$ элемента $a_{ij}$ a
соответствует номеру строки, второй индекс $j$ – номеру столбца, в котором расположен элемент $a_{ij}$.
Матрица $A$ может быть записана по формуле (\ref{eq:ref1}):

\begin{equation}
	A = \left(
	\begin{array}{cccc}
		a_{11} & a_{12} & \ldots & a_{1n} \\
		a_{21} & a_{22} & \ldots & a_{2n} \\
		\vdots & \vdots & \ddots & \vdots \\
		a_{m1} & a_{m2} & \ldots & a_{mn}
	\end{array}
	\right).
	\label{eq:ref1}
\end{equation}

\section{Стандартный алгоритм умножения матриц}
Произведением матрицы $A = (a_{ij})$, имеющей $m$ строк и $n$ столбцов, на матрицу $B = (b_{ij})$, имеющую
$n$ строк и $p$ столбцов, называется матрица $c_{ij}$, имеющая $m$ строк и $p$ столбцов, у которой элемент 
$C = (c_{ij})$ определяется по формуле (\ref{eq:ref2}):

\begin{equation}
	\begin{array}{cc}
		c_{ij} = \sum\limits_{r=1}^m a_{ir}b_{ri} & (i=1,2,\dots n; j=1,2,\dots p)
	\end{array}.
	\label{eq:ref2}
\end{equation}

\section{Алгоритм умножения матриц по Винограду}
Обозначим $i$ строку матрицы $А$ как  $\overline{u}$, $j$ столбец матрицы $В$ как $\overline{v}$ \cite{winograd-origin}. Тогда элемент $c_{ij}$ определяется по формуле (\ref{eq:ref3}):

\begin{equation}
	c_{ij} = \overline{u} \times \overline{v} =
	\begin{pmatrix} u_{1} & u_{2} & u_{3} & u_{4}\end{pmatrix}
	\begin{pmatrix} v_{1} \\ v_{2} \\ v_{3} \\ v_{4}\end{pmatrix} =
	u_{1}v_{1} + u_{2}v_{2} + u_{3}v_{3} + u_{4}v_{4}.
	\label{eq:ref3}
\end{equation}

Эту формулу можно представить в следующем виде (\ref{eq:ref4}):
\begin{equation}
	(u_{1}+v_{2})(u_{2}+v_{1}) + (u_{3}+v_{4})(u_{4}+v_{3}) - u_{1}u_{2} - u_{3}u_{4} - v_{1}v_{2} - v_{3}v_{4}
	\label{eq:ref4}
\end{equation}

На первый взгляд  может показаться, что выражение (\ref{eq:ref4}) задает больше работы, чем первое: вместо четырех умножений насчитывается их шесть, а вместо трех сложений -- десять. Выражение в правой части формулы  можно вычислить заранее и затем повторно использовать. На практике это означает, что над предварительно обработанными элементами придется выполнять лишь первые два умножения и последующие пять сложений, а также дополнительно два сложения.

\section{Алгоритм Штрассена}
Алгоритм Штрассена основан на принципе <<разделяй и властвуй>>, который позволяет уменьшить время выполнения умножения матриц за счет рекурсивного разделения на матрицы меньшего размера \cite{shtrassen}. Пусть $A$ и $B$ -- две матрицы размера $n \times n$, где $n$ -- степень числа 2. Каждую матрицу $A$ и $B$ можно разбить на четыре матрицы размером $n/2 \times n/2$ и через них выразить произведение матриц $A$ и $B$ (\ref{eq:shtrassen}):

\begin{equation}
	\label{eq:shtrassen}
	C = \begin{pmatrix}
		C_{11} & C_{12} \\
		C_{21} & C_{22} \\
	\end{pmatrix} = \begin{pmatrix}
		A_{11} & A_{12} \\
		A_{21} & A_{22} \\
	\end{pmatrix}
	\begin{pmatrix}
		B_{11} & B_{12} \\
		B_{21} & B_{22} \\
	\end{pmatrix},
\end{equation}
где элементы матрицы $C$ и их компоненты вычисляются как (\ref{eq:components_shtrassen}): 
\begin{equation}
	\label{eq:components_shtrassen}
	\begin{split}
	C_{11} = A_{11}B_{11} + A_{12}B_{21}, \\ 
	C_{12} = A_{11}B_{12} + A_{12}B_{22}, \\ 
	C_{21} = A_{21}B_{11} + A_{22}B_{21}, \\
	C_{22} = A_{21}B_{12} + A_{22}B_{22}.
	\end{split}
\end{equation}
Такой метод требует 8 умножений для вычисления $C_{ij}$ в классическом произведении. Алгоритм Штрассена заключается в наборе из семи новых матриц $M_{i},\ i=\overline{1;7}$, которые используются для выражения $C_{ij}$ с помощью \text{7 умножений (\ref{eq:opt_shtrassen}):}

\begin{equation}
	\label{eq:opt_shtrassen}
	\begin{aligned}
		M_{1} & = (A_{11}+A_{12})(B_{11}+B_{12}), \\ 
		M_{2} & = (A_{21}+A_{22})B_{11}, \\ 
		M_{3} & = A_{11}(B_{12}-B_{22}), \\
		M_{4} & = A_{22}(A_{21}-B_{11}), \\
		M_{5} & = (A_{11}+A_{12})B_{22}, \\ 
		M_{6} & = (A_{21}-A_{11})(B_{11}+B_{12}), \\ 
		M_{7} & = (A_{12}-A_{22})(B_{21}+B_{22}). \\
	\end{aligned},
\end{equation}
Тогда выражения $C_{ij}$ вычисляются как (\ref{eq:res_shtrassen}):
\begin{equation}
	\label{eq:res_shtrassen}
	\begin{aligned}
		C_{11} & = M_{1} + M_{4} - M_{5} + M_{7}, \\ 
		C_{12} & = M_{3} + M_{5}, \\ 
		C_{21} & = M_{2} + M_{4}, \\
		C_{22} & = M_{1} - M_{2} + M_{3} + M_{6}. \\
	\end{aligned}
\end{equation}

\section{Оптимизации алгоритмов}
При реализации алгоритмов умножения матрицы возможны следующие методы оптимизации:
\begin{enumerate}
	\item Предвычисление некоторых слагаемых для алгоритма.
	\item Замена операции $x = x + k$ на $x \mathrel{+}= k$.
	\item Замена умножения на 2 на побитовый сдвиг.
\end{enumerate} 

\section*{Вывод}
В данном разделе были рассмотрены алгоритмы умножения матриц стандартным образом, по Винограду, по Штрассену, а также приведены соответствующие математические расчеты. Предложены методы оптимизации расчетов.