\chapter{Технологический раздел}
В данном разделе приводятся требования к программному обеспечению, выбор средств реализации и модули программы.
\section{Требования к программному обеспечению}
Программа должна отвечать следующим требованиям:
\begin{itemize}
	\item на вход программе подаются два массива сгенерированных целых чисел, корректный размер которых задает пользователь;
	\item осуществляется выбор алгоритма умножения матриц из меню;
	\item на вход программе подаются только корректные данные;
	\item на выходе программа выдает результат -- матрицу, полученную в результате умножения двух входных.
\end{itemize}

\section{Выбор средств реализации}
Для реализации алгоритмов был выбран язык программирования \text{Python \cite{python}}. При замере процессорного времени используется модуль time с функцией process\_time \cite{process_time}.

\section{Модули программы}
\subsection{Стандартное умножение матриц}
Реализация умножения матриц стандартным образом приведена на листинге \ref{lst:ordinary_multiply}.
\lstinputlisting[label=lst:ordinary_multiply, caption=Стандартное умножение матриц, basicstyle=\scriptsize, numbers=none]{lst/ordinary_multiply.py}
\subsection{Умножение матриц по Винограду стандартный}
Реализация умножения матриц по Винограду приведена на листинге \ref{lst:vinograd}.
\lstinputlisting[label=lst:vinograd, caption=Умножение матриц по Винограду, basicstyle=\scriptsize, numbers=none]{lst/vinograd.py}
\subsection{Умножение матриц по Винограду оптимизированный}
Реализация оптимизированного умножения матриц по Винограду приведена на листинге \ref{lst:optimized_vinograd}.
\newpage
\lstinputlisting[label=lst:optimized_vinograd, caption=Умножение матриц по Винограду (оптимизированный), basicstyle=\scriptsize, numbers=none]{lst/optimized_vinograd.py}
\subsection{Умножение матриц Штрассена}
Реализация умножения матриц Штрассена приведена на листинге \ref{lst:strassen}.
\newpage
\lstinputlisting[label=lst:strassen, caption=Умножение матриц Штрассена, basicstyle=\scriptsize, numbers=none]{lst/strassen.py}
Вызов рекурсивной функции Штрассена приведена на листинге \ref{lst:strassen_recursive} и \ref{lst:strassen_recursive_2}.
\lstinputlisting[label=lst:strassen_recursive, caption=Вызов рекурсивной функции Штрассена. Часть 1, basicstyle=\scriptsize, numbers=none]{lst/strassen_recursive.py}
\lstinputlisting[label=lst:strassen_recursive_2, caption=Вызов рекурсивной функции Штрассена. Часть 2, basicstyle=\scriptsize, numbers=none]{lst/strassen_recursive_2.py}
Матричное сложение и вычитание приведены на листинге \ref{lst:matrix_ops}.
\lstinputlisting[label=lst:matrix_ops, caption=Матричное сложение и вычитание, basicstyle=\scriptsize, numbers=none]{lst/matrix_ops.py}

\section{Тестирование}
Для тестирования используется метод черного ящика. В данном разделе приведена таблица \ref{table:ref1}, в которой указаны классы эквивалентностей тестов.
\begin{table}[H]
	\centering
	\captionsetup{singlelinecheck = false, justification=raggedleft}
	\caption{Таблица тестов}
	\label{table:ref1}
	\begin{tabular}{|c|c|c|c|c|}
		\hline
		№ &Описание теста & Матрица 1  &  Матрица 2   &  Ожидаемый результат\\\hline
		1& Квадратный размер  & $\begin{pmatrix}6 & 9 & 8\\0 & 3 & 6\\4 & 9 & 5\end{pmatrix}$
		& $\begin{pmatrix}9 & 6 & 0\\2 & 3 & 5\\6 & 8 & 7\end{pmatrix}$
		& $\begin{pmatrix}120 & 127 & 101\\42 & 57 & 57\\84 & 91 & 80\end{pmatrix}$
		\\ \hline
		2& Разный размер	  & $\begin{pmatrix}2 & 3 & 2\\3 & 9 & 2\end{pmatrix}$
		& $\begin{pmatrix}2 & 4\\1 & 7\\4 & 1\end{pmatrix}$
		& $\begin{pmatrix}15 & 31\\23 & 77\end{pmatrix}$
		\\ \hline
		3&Разный размер		& $\begin{pmatrix}5 & 1\\0 & 4\\6 & 4\end{pmatrix}$
		& $\begin{pmatrix}2 & 1 & 7\\4 & 6 & 1\end{pmatrix}$
		& $\begin{pmatrix}14 & 11 & 36\\16 & 24 & 4\\28 & 30 & 46\end{pmatrix}$
		\\ \hline
		\multirow{2}{*}{4}  &Неподходящий
		& \multirow{2}{*}{$\begin{pmatrix}5 & 1\\0 & 4\end{pmatrix}$}
		& \multirow{2}{*}{$\begin{pmatrix}5 & 1\\0 & 4\end{pmatrix}$}
		& Несоответствие\\
		&размер & & &размеров.
		\\ \hline
	\end{tabular}
\end{table}
Для алгоритмов умножения матриц стандартным образом, по Винограду и оптимизированный по Винограду, а также Штрассена данные тесты были пройдены успешно.
\section*{Вывод}
В данном разделе был выбран язык программирования, средства замера процессорного времени. Реализованы функции, описанные в аналитическом разделе, и проведено их тестирование методом черного ящика по таблице \ref{table:ref1}. 