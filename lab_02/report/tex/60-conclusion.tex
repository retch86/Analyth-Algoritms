\chapter*{Заключение}
\addcontentsline{toc}{chapter}{Заключение}
В ходе выполнения лабораторной работы были рассмотрены алгоритмы умножения матриц стандартным способом, по Винограду,  Штрассена. Проведена оптимизация алгоритма Винограда. Выполнено описание каждого из этих алгоритмов, приведены соотвествующие математические расчёты трудоемкости каждого из них.

При анализе временных характеристик каждого из этих алгоритмов можно сделать следующие выводы: при помощи оптимизации алгоритма Винограда удалось уменьшить трудоемкость стандартного способа умножения матриц в среднем на 47\% при нечетных размерах, и в среднем на 40\% при четных. Это решение позволит выполнять вычисления на количестве данных порядка тысячи и выше быстрее стандартного способа. 

Умножение матриц при помощи алгоритма Штрассена увеличивает время выполнения операции в среднем в 8-10 раз стандартного способа и алгоритма Винограда. 

Из набора алгоритмов умножения матриц <<стандартный>>, <<по Винограду>>, <<по Винограду оптимизированный>>, <<Штрассена>> необходимо использовать оптимизированную версию Винограда.