\part*{ВВЕДЕНИЕ}
\addcontentsline{toc}{part}{\textbf{ВВЕДЕНИЕ}}

Матричное умножение лежит в основе нейронных сетей. Большинство операций при обучении нейронной сети требуют перемножение матриц. Для этого требуется высокая скорость вычислений.

Стандартный алгоритм умножения матриц на больших данных, исчисляемых миллиардами, выполняет вычисления не самым быстрым способом. Существуют различные оптимизации. Одной из них является алгоритм Винограда, который позволяет сократить время вычислений \cite{winograd-origin}. На практике алгоритм Копперсмита—Винограда не используется, так как он имеет очень большую константу пропорциональности и начинает выигрывать в быстродействии у других известных алгоритмов только для тех матриц, размер которых превышает память современных компьютеров.

Целью лабораторной работы является изучение и реализация алгоритмов умножения матриц. Для её достижения необходимо выполнить следующие задачи:
\begin{itemize}
	\item выполнить анализ алгоритмов умножения матриц стандартным способом, по Винограду и Штрассена;
	\item формально описать данные алгоритмы;
	\item выполнить оптимизацию алгоритма Винограда;
	\item реализовать алгоритмы умножения матриц;
	\item выполнить тестирование реализации алгоритмов методом черного ящика;
	\item провести сравнительный анализ этих алгоритмов по процессорному выполнению времени на основе экспериментальных данных.
\end{itemize}